\section{Introduction}

Reinforcement Learning is a part of Machine Learning, where an agent is trained to interact in a desired way with its environment. Based on the current state, the agent decides for an action and can receive a reward for the chosen action. \cite{Salvador2020}

The potential of Reinforcement Learning is proven many times in varying contexts. E.g. attention was generated by the success of DeepMind's AlphaGo, the first artificial agent defeating a human in the game Go. For training this agent Silver et al. used Reinforcement Learning. \cite{Silver1140}

Salvador, Oliveira and Breternitz have summarized the evolution of Rein\-force\-ment learning in a literature review \cite{Salvador2020}. It begins in 1989 with the publication introducing Q-learning. In the recent past many publications deal with the combination of Deep Learning with Reinforcement Learning. This area is known as Deep Q-learning.

As part of this project Reinforcement Learning is used for learning how to play Bomberman. Bomberman is a strategic board game, which is played on a field containing walls, crates, bombs, coins and other players. For winning the game one must gather points by killing other players and collecting coins. Typically in each game round first the walls around the player have to be removed by placing bombs. Next the agent can navigate towards coins or towards his opponents and kills them by placing bombs. \cite{Kormelink2018}

After introducing the fundamentals and related work in chapter \ref{fundamentals_related_work}, the approach is described in chapter \ref{approach}. Therefore the selected Reinforcement Learning method and the training process are presented. In chapter \ref{experimental_results} the results of the experiments are described. A conclusion is drawn in chapter \ref{conclusion}.